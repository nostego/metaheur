\documentclass{report}
\usepackage[francais]{babel}
\usepackage[utf8]{inputenc}
\usepackage[T1]{fontenc}
\usepackage{float}
\usepackage{amssymb}
\usepackage{stmaryrd}

\usepackage{graphicx}
\usepackage{subfig}

\makeatletter
\newcommand{\Spvek}[2][r]{%
  \gdef\@VORNE{1}
  \left(\hskip-\arraycolsep%
    \begin{array}{#1}\vekSp@lten{#2}\end{array}%
  \hskip-\arraycolsep\right)}

\def\vekSp@lten#1{\xvekSp@lten#1;vekL@stLine;}
\def\vekL@stLine{vekL@stLine}
\def\xvekSp@lten#1;{\def\temp{#1}%
  \ifx\temp\vekL@stLine
  \else
    \ifnum\@VORNE=1\gdef\@VORNE{0}
    \else\@arraycr\fi%
    #1%
    \expandafter\xvekSp@lten
  \fi}
\makeatother

\newcommand{\HRule}{\rule{\linewidth}{0.5mm}}
\bibliographystyle{unsrt}
\begin{document}

\begin{titlepage}

\begin{center}

\textsc{\LARGE Rapport}\\[1.5cm]

\textsc{\Large EPITA}\\[0.5cm]

\HRule \\[0.4cm]
{ \huge \bfseries Métaheuristiques pour l'optimisation difficile}\\[0.4cm]

\HRule \\[1.5cm]

\large
\emph{Auteurs:}\\
Loïc \textsc{Bethmont}\\
Victor \textsc{Lenoir}\\

\vfill

% Bottom of the page
{\large \today}

\end{center}

\end{titlepage}
%\maketitle
\newpage
\tableofcontents
\newpage

\chapter{Introduction}

Une métaheuristique est un algorithme d'optimisation permettant de
résoudre des problèmes difficiles de manière simple.  Un problème est
dis difficile quand il est inconcevable de parcourir la totalité de
l'espace de solutions pour trouver l'optimum global.


Les métaheuristiques sont des algorithmes très géneraux et avec un
haut niveau d'abstraction et peuvent par conséquent être appliquées sur
un grand nombre de problèmes différents.


Ce sont géneralement des méthodes stochastiques itératives qui tentent
de converger vers un optimum global.


Les métaheuristiques sont très utilisées lorsqu'un problème est trop
complexe pour obtenir une solution analytique ou une autre solution
algorithmique ou lorsqu'une solution alternative est trop coûteuse ou
encore lorsqu'on ne connaît pas de méthodes pour résoudre le problème.

\chapter{Algorithmes}

\section{Recuit simulé}

\section{Programmation génétique}

\chapter{Benchmarks}

\chapter{Conclusion}

\end{document}
